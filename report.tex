\documentclass[11pt]{article}
\usepackage[a4paper, margin=2cm]{geometry}
\usepackage{fontspec}
\setmainfont{Arial}

\usepackage{fancyhdr}
\pagestyle{fancy}
\fancyhead[R]{Student Number: \textbf{170266442}}

\usepackage[backend=biber, style=numeric, citestyle=nature]{biblatex}
\addbibresource{epm307AA.bib} %Imports bibliography file

\usepackage{titlesec}
\titlespacing\section{0pt}{*0}{*0}

\setlength\parindent{0pt}
\setlength{\parskip}{1em}

\begin{document}

\begin{center}
    \LARGE\textbf{Preventing Sudden Infant Death in New Zealand}
\end{center}

\section{Definitions}

When examining the incidence of, and proposing methods to prevent sudden infant death, it is vital to clearly define the problem. 
In this brief we will make the distinction between two relevant diagnoses: sudden infant death syndrome (SIDS) and sudden unexpected death in infancy (SUDI). 
SIDS was historically used in the literature, being synonymous with “cot death”, and is a diagnosis of exclusion, where the cause of death is unexplained.
SUDI has been adopted more recently, and is a broader definition with an International Classification of Diseases 10th version (ICD-10) code set being proposed to include determined but unexpected causes of death in infancy; such as asphyxia during sleep, unsafe sleeping, and unexplained causes of sudden death (SIDS) \autocite{Mitchellcombinationbedsharing2017}.
Whilst there are similarities between the two, SUDI is preferable as a more thorough clinical investigation may contribute towards an explainable diagnosis, which would otherwise not be possible with SIDS.
Throughout this report, all confidence intervals are displayed as 95\% confidence intervals, unless stated otherwise.

\section{Public Health Burden and Temporal Change}

Between November 1987 and October 1990, SIDS accounted for approximately 68\% of all post-neonatal (between 28 days and 1 year of life) deaths in New Zealand \autocite{MitchellFourmodifiableother1992}, with an average rate of 3.53 deaths per 1000 live births. 
The next largest cause of post-neonatal deaths were congenital disorders, at a much lower incidence of 0.78 deaths per 1000 live births. 
Research suggests that SIDS incidence in New Zealand was considerably higher than comparable countries \autocite{MitchellFourmodifiableother1992}, and therefore not an anomaly of the study. 
As further studies have been conducted, producing a greater understanding of the modifiable risk factors for SIDS and SUDI, public health interventions have been enacted, and both the frequency and incidence rate of SIDS and SUDI have steadily fallen (Fig 1). 
However, this decline plateaued in 2011 at higher rates than comparable countries, with a peak age of death of 1 – 3 months, similar to that observed by SWISS study of southwest England \autocite{Mitchellcombinationbedsharing2017}.

\section{Risk Factors}

Mitchell et al \autocite{Mitchellcombinationbedsharing2017,MitchellFourmodifiableother1992} identified a number of modifiable risk factors for SIDS and SUDI, the most notable being maternal smoking during pregnancy, infant sleeping position, parental bed sharing with the infant, and breastfeeding, that were all acknowledged in both studies. 

Maternal smoking was the most important risk factor identified in both studies, where the odds of SIDS (1992) and SUDI (2017) death was 4.09 ( and 5.28 higher in infants of mothers who smoked during pregnancy, compared to infants of non-smokers. 
After adjusting for confounding variables, such as ethnic group, marital status, and birthweight, the odds of SUDI deaths increased to 6.01 (CI

Infant sleeping position

In addition to acting as an individual risk for SIDS and SUDI, maternal smoking and bed sharing heavily interacted to increase the risk posed to the infant. 


\section{Prevention Programmes}

\section{Future Directions}

\section{Future Research}

\printbibliography

\end{document}
